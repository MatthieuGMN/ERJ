\title{Physiotherapy management in patients with Covid-19}

\author[1,2,3]{Clément Medrinal}
\author[1,4]{Yann Combret}
\author[5,6]{Tristan Bonnevie}
\author[6]{Pauline Smondack}
\author[5,6]{Francis-Edouard Gravier}
\author[7]{Marius Lebret}
\author[1,4]{Guillaume Prieur}

\affil[1]{Groupe Hospitalier du Havre, Intensive Care Unit department and Pulmonology, avenue Pierre Mendes France 76290 Montivilliers, France.}
\affil[2]{Université Paris-Saclay, UVSQ, Erphan, 78000, Versailles, France.}
\affil[3]{Institut de Formation en Masso-kinésithérapie Saint-Michel, 75015 Paris, France}
\affil[4]{Institut de Recherche Expérimentale et Clinique (IREC), Pôle de Pneumologie, ORL \& Dermatologie, Groupe de Recherche en Kinésithérapie Respiratoire, Université Catholique de Louvain, 1200 Brussels, Belgium.}
\affil[5]{Normandie Univ, UNIROUEN, EA3830-GRHV, 76 000 Rouen, France; Institute for Research and Innovation in Biomedicine (IRIB), 76 000 Rouen, France.}
\affil[6]{ADIR Association, 76230 Bois-Guillaume, France.}
\affil[7]{Institut Universitaire de Cardiologie et de Pneumologie de Québec (IUCPQ), groupe de recherche en hypertension pulmonaire, 2725 Ch Ste-Foy, Québec, QC G1V 4G5.}

\keywords{Covid-19; critically ill; intensive care unit; physiotherapy; rehabilitation}
