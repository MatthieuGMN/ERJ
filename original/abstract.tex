\\
\textbf{Background}: The Keele STarT MSK Tool is a 10-item questionnaire developed to classify patients suffering from one of the five most common types of musculoskeletal pain into three sub-groups according to their risk of chronic pain (i.e. low risk, medium risk and high risk).
\textbf{Objectives}: The objective of the present study was to translate the Keele STarT MSK Tool into French and to evaluate its main psychometric properties.
\textbf{Methods}: The translation and intercultural adaptation of the questionnaire were carried out using a 6-step process. The following psychometric properties were investigated: floor and ceiling effects, construct validity, internal consistency and test-retest reliability including Standard Error of Measurement and Smallest Detectable Change. 
\textbf{Results}: 101 patients suffering from musculoskeletal pain participated in the study. No floor nor ceiling effects were observed. A Cronbach’s alpha of 0.65 was found, revealing moderate internal consistency. All items were demonstrated to be significantly correlated with the total score (range of correlations: r=0.2 for item 7 to r=0.78 for item 1). A significant correlation of r=0.78 between the French Keele STarT MSK Tool and the ÖMPSQ-short was found. Nevertheless, a poor agreement between tools was found, highlighted by a Kappa value of 0.57. Test-retest reliability was excellent (Intraclass Correlation Coefficient 0.97). The Standard Error of Measurement and Smallest Detectable Change of ±1.17 were 0.42 and ±1.17, respectively. \textbf{Conclusion}: A validated French version of the Keele STarT MSK Tool is now available and can be used by health practitioners to stratify patients as being low, medium or high risk for persistent musculoskeletal pain.  