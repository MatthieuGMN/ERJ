\\
\textbf{Background}: Unintentional leak (UL) is a frequent adverse effect in patients treated with CPAP. We previously published a novel methodology for analyzing the determinants of UL using polysomnography. In the present study, we proposed a simplified recording system using a type-3 polygraphic device (Somnolter®).
\\
\textbf{Objectives}: 1)to describe the individual determinants of UL provided by the Somnolter® software in OSA patients treated with automatic-CPAP; 2)to describe, based on this analysis, the clinical consensus of 4 physicians on how to manage individual UL situations.
Methods: Somnoler® recordings performed under nasal automatic-CPAP were automatically analyzed with the APIOS software. For each polygraphic recording, APIOS provided the odds ratio and the confidence intervals of the potential determinants of UL: mouth opening, CPAP pressure, body position and mandibular oscillation. Based on these results, each of four physicians was asked to choose one of four strategies: (i)increase/decrease therapeutic pressure; (ii)change nasal mask for oro-nasal mask/chinstrap; (iii)favor a non-supine/supine position; (iv)no action, for individual leak management. Subsequently, a meeting was held to determine a consensus choice for each individual case.
\\
\textbf{Results}: 78 consecutive patients underwent a home-polygraphy with Somnolter®. Fifty recordings were analyzed (16 females; 65[57; 75]years; BMI=31.1[27.4; 35.3]kg/m2). Individual diagnosis of UL was routinely feasible. The determinants of UL were heterogeneous in the population Based on the results from this analysis, we established consensus leak management strategies at the individual level. The average Cohen $\kappa$ coefficient for the 4 raters was 0.58. Pressure modification was proposed in 36% of patients, no action in 24%, installation of a facial mask/chinstrap in 22% and positional treatment in 18%.
\\
\textbf{Conclusion}: the use of type-3 polygraphy for characterizing leak determinants in patients treated with nasal automatic-CPAP is feasible in routine practice. Leak determinants are patient specific. Inter-rater concordance for determining individual leak management strategies demonstrated a “fair” level of agreement.
\bigbreak
\underline{Clinical trial registration}: NCT03381508